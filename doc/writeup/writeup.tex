\documentclass{article}
\usepackage[T1]{fontenc}
\usepackage[utf8x]{inputenc}
\usepackage{amsmath,amssymb,amsthm}
\usepackage{mathtools}
\usepackage{hyperref}
\usepackage{enumerate}
\usepackage{semantic}
\usepackage{tikz-qtree}
\usepackage{titling}
\setlength{\droptitle}{-7em}
\usepackage{gb4e}
\noautomath

\newtheorem{mydef}{Definition}

\begin{document}


%%%%%%%%%%%%%%%%%%%%%%%%%%%%%%%%%%%%%%%%%%%%%%%%%%%%%%%%%%%%%%%%%%%%%%%%
%                                                         FRONT MATTER %
%%%%%%%%%%%%%%%%%%%%%%%%%%%%%%%%%%%%%%%%%%%%%%%%%%%%%%%%%%%%%%%%%%%%%%%%


\title { Adjunction in Minimalism                                      }
\author{ Patrick Niedzielski \\
  \href{mailto:pmn25@cornell.edu}{\texttt{pmn25@cornell.edu}}          }
\maketitle

% \begin{abstract}
%   The problem of adjunction has plagued Syntactic Theory since the
%   adoption of Minimalism in the mid-90s.  Although the empirical
%   properties of adjuncts are well understood, no analysis has been
%   ever been widely accepted.  In this paper, we briefly discuss the
%   syntactic properties of adjuncts and how early attempts fail to
%   capture these properties.  We then discuss three recent attempts to
%   account for adjunction within Minimalist frameworks, and then
%   present a critical analysis of these accounts.
% \end{abstract}


%%%%%%%%%%%%%%%%%%%%%%%%%%%%%%%%%%%%%%%%%%%%%%%%%%%%%%%%%%%%%%%%%%%%%%%%
%                                                          MAIN MATTER %
%%%%%%%%%%%%%%%%%%%%%%%%%%%%%%%%%%%%%%%%%%%%%%%%%%%%%%%%%%%%%%%%%%%%%%%%


%%%%%%%%%%%%%%%%%%%%%%%%%%%%%%%%%%%%%%%%%%%%%%%%%%%%%%%%%%%%%%%%%%%%%%%%
\section{Introduction}
\label{sec:introduction}

Since the general acceptance of the Minimalist Program, and with it
Bare Phrase Structure theory, Syntactic Theory has struggled to
explain the common and empirically well-understood phenomenon of
\textit{adjunction} of two constituents.  Adjunction is not a
corner-case of syntax that can be easily ignored, but it does not seem
simple to fit it into the existing \textbf{Merge} and \textbf{Move}
machinery posited in most Minimalist frameworks.  This, combined with
their transparency in further applications of \textbf{Merge} and
\textbf{Move}, makes them a long-standing problem in Minimalism.
Chametzky \cite{chametzky2003} uses this as an argument for the
(undoubtedly intentionally provocative) claim that ``adjuncts have no
syntax.''  We do not agree with the strength of this claim, and we
will attempt to show that current research does indeed indicate that
adjuncts can play well with Minimalist machinery.

In this paper, we present a critical discussion of several recent
Minimalist analyses of adjunction.  We first present a minimal set of
syntactic properties of adjuncts that any account of adjunction should
be able to explain.  These properties are widely accepted, and we
present them here for exposition, as background for our analysis.  We
then briefly discuss early Minimalist attempts to explain these
properties and why they are inadequate.  Then, we present three recent
accounts of adjunction within Minimalist frameworks---Meaghan Fowlie's
\textit{Minimalist Grammars with Adjunction} (MGAs) \cite{fowlie2015},
Norbert Hornstein's label-less adjunction account
\cite{hornstein2008b}, and Tim Hunter's arguments and adjunction
account \cite{hunter2011}---and analyze the benefits and drawbacks of
each approach.  Finally, we will present some initial thoughts on
where further study could be directed to draw from the benefits of all
three approaches.

During the course of this research, to gain more experience with
Fowlie's MGA formalism, we implemented a CKY-like chart parser for
MGAs.  We also discuss the high-level design of this chart parser and
then discuss the process of implementing it in Appendix
\ref{sec:implementation}.


%%%%%%%%%%%%%%%%%%%%%%%%%%%%%%%%%%%%%%%%%%%%%%%%%%%%%%%%%%%%%%%%%%%%%%%%
\section{Syntactic Properties of Adjunction}
\label{sec:adjunction-prop}

Despite the general lack of consensus on an adequate and elegant
treatment of adjunction in Minimalist theories of syntax, there is
little contention about the empirical syntactic properties that
adjuncts display.  We will briefly consider these properties here
before looking at several accounts of adjuncts that attempt to capture
and explain these properties.

The basic set of syntactic properties of adjunction is discussed in
\cite{fowlie2015} and is summarized here. Syntactically, adjuncts are
subconstituents that are \textit{optional}, \textit{iterable}, and
\textit{transparent to selection}.  Furthermore, adjuncts have a
category, can be complex syntactic objects, and can only adjoin to
target phrases of certain other, determined by the adjunct's category.

\begin{exe}
  \ex[]{The (small) cat. \hfill \textbf{Optionality}}
  \ex[]{The small, small, small cat. \hfill \textbf{Iterability}}
  \ex[]{
    \begin{xlist}
    \ex[]{\atcenter{
          \arrowalign{[\textsubscript{DP}&\ A&\ [small&\ cat.]]\cr
            & \fillright\vrule&\lf&\fillleft\pu\cr}
        }}
    \ex[]{\atcenter{
          \arrowalign{[\textsubscript{DP}&\ $\varnothing$&\ [small&\ cats.]]\cr
            & \fillright\vrule&\lf&\fillleft\pu\cr}
        }}
    \ex[*]{\atcenter{
          \arrowalign{[\textsubscript{DP}&\ A&\ [small&\ cats.]]\cr
            & \fillright\vrule&\lf&\fillleft\pu\cr}
        } \hfill \textbf{Transparency to selection}}
    \end{xlist}
  }
  \ex[]{
    \begin{xlist}
    \ex[]{The [\textsubscript{NP} [\textsubscript{AP} happy] boy]}
    \ex[*]{He [\textsubscript{VP} [\textsubscript{AP} happy] runs]
      \hfill \textbf{Target Categories}}
    \end{xlist}
  }
\end{exe}

These attributes distinguish adjunction from a feature-driven
\textbf{Merge} between heads and arguments, à la Chomsky
\cite{chomsky1995}.  Adjuncts are never required\footnote{Although
  this has been disputed by Grimshaw and Vikner \cite{grimshaw1993},
  who give the following data as examples of a contrast they claim
  shows obligatory adjuncts:

  \begin{exe}
    \ex
    \begin{xlist}
    \ex[\#]{This house was built.}
    \ex[]{This house was built last year.}
    \end{xlist}
    \ex
    \begin{xlist}
    \ex[\#]{A built house.}
    \ex[]{A recently built house.}
    \end{xlist}
  \end{exe}

  Goldberg and Ackerman \cite{goldberg2001} present a unified
  pragmatic account of this data, as well as seemingly obligatory
  adjuncts in middle voice constructions, arguing that the adjuncts
  are not syntactically obligatory, and the contrast is due to
  pragmatic contexts.}, unlike arguments which must be present to
discharge $\theta$-roles and other selectional requirements.
Similarly, once the selectional requirements of a head are met, there
can be no further \textbf{Merge}s or \textbf{Move}s to satisfy them
again, meaning that arguments, unlike adjuncts, are not iterable.
Finally, when a head or projection selects for another syntactic
object as an argument, it becomes the head of the new projection;
adjuncts, however, which seem to ``select for'' another target
syntactic object, do not become heads of the resulting structure,
making them transparent to further \textbf{Merge}s and \textbf{Move}s.
Hornstein and Nunes \cite{hornstein2008a} also give a similar set of
requirements which describe the same empirical phenomena, assuming a
GB or Minimalist framework, but not committing to any specific account
of adjunction.% According to Hornstein and Nunes, adjunction:

% \begin{enumerate}[(a)]
% \item \label{hn-adj-prop-bar}
%   preserves bar-level information of the target,
% \item \label{hn-adj-prop-cat}
%   preserves categorial information of the target,
% \item \label{hn-adj-prop-head}
%   preserves headedness of the target,
% \item \label{hn-adj-prop-recurse}
%   and has no upper bound on the number of times it can apply.
% \end{enumerate}

% These requirements imply directly optionality, iterability, and
% transparency to selection.  Adjuncts are also necessarily iterable
% under these requirements, as labeling information of the target is
% preserved in the resulting adjunction structure by
% (\ref{hn-adj-prop-bar}) and (\ref{hn-adj-prop-cat}), yielding a
% structure that looks the same to further adjunction operations---of
% which there may be an unbounded number, by
% (\ref{hn-adj-prop-recurse}).  Finally, because (\ref{hn-adj-prop-bar})
% and (\ref{hn-adj-prop-cat}) require the resulting adjunction structure
% to have the same label as the target of adjunction, and because
% (\ref{hn-adj-prop-head}) requires the $\phi$-features of the head to
% propagate up to the new adjunction structure, the resulting structure
% looks the same to further \textbf{Merge} and \textbf{Move} operations
% as well, making it transparent to selection.

In addition to the above core properties of adjunction, Fowlie
\cite{fowlie2015} also notes that adjuncts may be either unordered or
strictly ordered with one another, giving the following data as an
examples:

\begin{exe}
  \ex
  \begin{xlist}
    \ex[]{ The alliance officer shot Kaeli [\textsubscript{PP} in the
      cargo hold] [\textsubscript{PP} with a gun]. }
    \ex[]{ The alliance officer shot Kaeli [\textsubscript{PP} with a
      gun] [\textsubscript{PP} in the cargo hold]. \\
      \hspace*{\fill} \textbf{Unordered adjuncts} }
  \end{xlist}
  \ex
  \begin{xlist}
    \ex[]{ \label{adjunct-order-normal} Wear the enormous ugly green hat. \\
      \textit{Wear the hat that is enormous, ugly, and green.} }
    \ex[\#]{ \label{adjunct-order-marked} Wear the ugly enormous green hat. \\
      \textit{Of your enormous green hats, wear the ugly one.}  \\
      \hspace*{\fill} \textbf{Strictly ordered adjuncts} }
    \end{xlist}
\end{exe}

Fowlie further claims that there is a cross-linguistic hierarchy for
strictly ordered adjuncts, corresponding to classes such as ``Size''
and ``Goodness.''


%%%%%%%%%%%%%%%%%%%%%%%%%%%%%%%%%%%%%%%%%%%%%%%%%%%%%%%%%%%%%%%%%%%%%%%%
\section{Approaches to Adjunction}
\label{sec:approaches}

In this section, we will briefly present a some early approaches to
adjunction in Minimalist frameworks, and then present three recent
Minimalist accounts of adjunction.


%%%%%%%%%%%%%%%%%%%%%%%%%%%%%%%%%%%%%%%%%%%%%%%%%%%%%%%%%%%%%%%%%%%%%%%%
\subsection{Early Approaches to Adjunction}
\label{sec:early-approaches}

% Pre-Minimalist approaches to adjuncts generally explained them as
% maximal projections that attach at the X$'$ level (or in some
% accounts, such as \cite{radford1988}, at any level) conserving
% bar-level and categorial information.  This approach naturally and
% straight-forwardly satisfies the criteria laid out by Hornstein and
% Nunes in \cite{hornstein2008a} and reproduced in Section
% \ref{sec:adjunction-prop} above.\footnote{Although it does run up
%   against some problems within X-bar Theory itself, namely the desire
%   that all projections project up from lexical items alone: does this
%   mean the lexical item must know how many adjuncts are going to
%   adjoin to it before the adjunctions happen?}

With the introduction of the Minimalist Program in the mid-90s, and
with the rejection of X-bar Theory in favor of Bare Phrase Structure
(BPS), this simple account of adjunction was no longer possible.  In
\textit{The Minimalist Program} \cite{chomsky1995}, Chomsky notes this
and presents a work-around within BPS to account for adjuncts:

\begin{quotation}
  \noindent
  But adjunction forms a different object.  \dots Therefore, there
  must be an object constructed from $K$ but with a label distinct
  from its head $H(K)$.  One minimal choice is the ordered pair
  $\langle H(K), H(K) \rangle$.  We thus take $L = \{ \langle H(K),
  H(K) \rangle, \{ \alpha, K \} \}.$
\end{quotation}

Chomsky leaves this line of reasoning with the above unfulfilling
conclusion, which seems neither minimal nor in-line with the empirical
properties we noted in Section \ref{sec:adjunction-prop}, failing to
capture transparency to selection elegantly.  Because the result of
adjunction has a different label than its target, both it and the
target can be selected for.  But this means any lexical item that
selects for the $H(K)$ must also select for
$\langle H(K), H(K) \rangle$---a duplication that hardly seems
motivated.

Another early account of adjuncts was done within the context of Ed
Stabler's Minimalist Grammars (MGs) \cite{stabler2001}, a Minimalist
mathematical formalism that uses feature-driven \textbf{Merge} and
\textbf{Move} as the two structure-building operations.  Within this
approach, presented in \cite{fowlie2015} as the ``traditional MG
solution,'' the adjuncts themselves are modified, instead of the
\textbf{Merge} operation.  Adjuncts of a category \texttt{X} are said
to have features \texttt{=XX}$\gamma$; this allows them to combine
with \texttt{X} and produce an \texttt{X} as a result.  In doing so,
though, they do not preserve the headedness of the original
\texttt{X}---instead, the adjunct itself becomes the head!
Furthermore, there is no labeling here of the adjunct itself, meaning
adjuncts cannot be selected for: without duplicating lexical entry for
the adjunct with different feature set, there is no way for the
adjunct to be a complement of any other phrase (as an adjective,
usually an adjunct, might be if it is in a predicate).

Both of these approaches suffer from intractable empirical or
theoretical problems, so we turn our attention to more recent accounts
of adjunction.


%%%%%%%%%%%%%%%%%%%%%%%%%%%%%%%%%%%%%%%%%%%%%%%%%%%%%%%%%%%%%%%%%%%%%%%%
\subsection{Fowlie's Minimalist Grammars With Adjunction}
\label{sec:fowlie-mga}

In her 2015 dissertation \cite{fowlie2015}, Meaghan Fowlie approaches
the problem of adjunction from a computational perspective, building
on MGs.  In order to account for the properties we discussed in
Section \ref{sec:adjunction-prop}, Fowlie introduces a new structure
building operation, \textbf{Adjoin}, whose definition is reproduced
here as Definition \ref{def:adjoin}.  This new operation adjoins
either an adjunct and its adjacent target, or a non-adjacent
adjunct-target pair that results from movement.

\begin{mydef}[Adjoin]
\label{def:adjoin}
  Let $s,t \in \Sigma$ be strings, $Y,X \in \mathbf{sel}$ be
  categories, $i, j, m, n \in \mathbb{N}$,
  $mvrs \in (\Sigma^{*} \times F^{*})^{*}$ be a mover list, and
  $\alpha, \beta \in F^{*}$.
  \begin{align*}
    & \mathbf{Adjoin}(
      \langle s, [\mathtt{X}, \mathtt{i}, \mathtt{j}]\alpha, mvrs \rangle,
      \langle t, [\mathtt{Y}, \mathtt{n}, \mathtt{m}]\beta \rangle
      ) \\
    = &\begin{dcases*}
      \langle ts, [\mathtt{X}, \mathtt{i}, \mathtt{n}]\alpha \rangle, mvrs
      & if $\mathtt{n} \geq \mathtt{j}$ and $\mathtt{Y} \in \mathbf{Ad}(\mathtt{X})$ and $\beta = \epsilon$ \\
      \langle s, [\mathtt{X}, \mathtt{i}, \mathtt{n}]\alpha \rangle, \langle t, \beta \rangle, mvrs
      & if $\mathtt{n} \geq \mathtt{j}$ and $\mathtt{Y} \in \mathbf{Ad}(\mathtt{X})$ and $\beta = \epsilon$
    \end{dcases*}
  \end{align*}
\end{mydef}

In order to prevent any category of phrase from adjoining to any other
category of phrase, Fowlie introduces a function
$\mathbf{Ad}(\mathtt{X})$ into the grammar that takes in the category
of the target phrase and returns a set of all categories of adjuncts
that can adjoin to the target phrase.  Then, \textbf{Adjoin} just
checks whether the adjunct's category is in $\mathbf{Ad}(target)$
before adjoining.  If, for instance, a grammar defines
$\mathbf{Ad}(\mathtt{N}) = \{ \mathtt{A, P} \}$, \textbf{Adjoin} can
adjoin adjectival and prepositional phrases to noun phrases, but no
other adjunctions are possible.

Finally, to account for both unordered adjuncts and adjuncts with
strict ordering, Fowlie expands the type of categories to include two
\textit{indices} that describe, respectively, what level in the
hierarchy this adjunct belongs to, and what level the last adjunct to
the phrase belonged to.  For example, if ``Size'' adjuncts are to be
ordered before ``Goodness'' adjuncts, we can let the index \texttt{1}
correspond to ``Size'' adjuncts and \texttt{2} to ``Color'' adjuncts.
Then, given the lexical entries
$\textit{wolf} :: [\mathtt{N}, \mathtt{0}, \mathtt{0}]$,
$\textit{bad} :: [\mathtt{A}, \mathtt{1}, \mathtt{0}]$, and
$\textit{big} :: [\mathtt{A}, \mathtt{2}, \mathtt{0}]$, we can
disallow the derivation shown in Figure \ref{fig:fowlie-order-bad},
which tries to adjoin a ``Goodness'' adjunct to a phrase that had
already been adjoined to a ``Size'' adjunct.

\begin{figure}[h]
  \centering

  \begin{tikzpicture}
      \Tree
      [.*[\texttt{N},\texttt{0},\textit{?}]
        [.[\texttt{A},\texttt{1},\texttt{0}] bad ]
        [.[\texttt{N},\texttt{0},\texttt{2}]
          [.[\texttt{A},\texttt{2},\texttt{0}] big ]
          [.[\texttt{N},\texttt{0},\texttt{0}] wolf ] ] ]
    \end{tikzpicture}

    \caption{A disallowed MGA derivation with strictly ordered
      adjuncts.  The final \textbf{Adjoin} is not allowed, as it tries
      to adjoin an adjunct with index \texttt{1} to a phrase that has
      already adjoined an adjunct with index \texttt{2}.}
  \label{fig:fowlie-order-bad}
\end{figure}

Fowlie shows that MGAs are weakly equivalent to MGs (that is, that
they describe the same set of languages).  MGAs are not strongly
equivalent to MGs, as they introduce a new operation, so their
derivations will not be isomorphic.


%%%%%%%%%%%%%%%%%%%%%%%%%%%%%%%%%%%%%%%%%%%%%%%%%%%%%%%%%%%%%%%%%%%%%%%%
\subsection{Hornstein's Label-less Adjuncts}
\label{sec:hornstein-adjuncts}

In his book \textit{A Theory of Syntax} \cite{hornstein2008b},
Hornstein approaches adjunction from a different, more theoretical
perspective.  Hornstein initially describes what he calls Darwin's
Problem, namely that the seemingly complex Faculty of Language (FL)
evolved in a very short amount of time on an evolutionary scale.
Hornstein lays out a plan to simplify FL by reducing complex
operations to simpler components and removing those which he considers
epiphenomenal, such as C-Command and the A-over-A Condition.  In
particular, Hornstein decomposes \textbf{Merge} into a sequence of two
operations, \textbf{Concatenate} (denoted by $\widehat{}\;$), which
takes two syntactic atoms and returns an ordered concatenate, and
\textbf{Label}, which takes a concatenate and prevents its
constituents from being reconcatenated, yielding a new syntactic atom.
An example of this process is shown in (\ref{hornstein-merge}).
Hornstein's claim is that of the fundamental operations he identifies,
\textbf{Label} is the recent evolutionary innovation.

\begin{exe}
  \ex[]{ \label{hornstein-merge}
    \begin{xlist}
      \ex{} \textit{see the ball} \hfill \textbf{Surface form}
      \ex{} the $\widehat{}$ ball \hfill \textbf{Concatenate}
      \ex{} [\textsubscript{the} the $\widehat{}$ ball] \hfill
      \textbf{Label}
      \ex{} see $\widehat{}$ [\textsubscript{the} the $\widehat{}$
      ball] \hfill \textbf{Concatenate}
      \ex{} [\textsubscript{see} see $\widehat{}$ [\textsubscript{the}
      the $\widehat{}$ ball]] \hfill \textbf{Label}
    \end{xlist}
  }
\end{exe}

Within this framework, Hornstein tackles the problem of adjunction,
arguing that it is actually a simpler operation than \textbf{Merge},
being composed of a \textbf{Concatenate} and then an \emph{optional}
\textbf{Label}.  The possible structures for the surface form
\textit{run in the house} and \textit{run in the house with scissors}
under this story are given in
(\ref{hornstein-adjoin1}--\ref{hornstein-adjoin2}).  By allowing
adjuncts to be part of either labeled or unlabeled structures,
Hornstein is able to explain how adjuncts can either be pied-piped or
stranded during movement as simply a structural ambiguity in the
derivation, such that stranded adjuncts are necessarily unlabeled.

\begin{exe}
  \ex[]{ \label{hornstein-adjoin1}
    \begin{xlist}
      \ex{} run $\widehat{}$ in-the-house
      \ex{} [\textsubscript{run} run $\widehat{}$ in-the-house]
    \end{xlist}
  }
  \ex[]{ \label{hornstein-adjoin2}
    \begin{xlist}
      \ex{}
      \setlength{\tabcolsep}{0.2em}
      \begin{tabular}{rl}
        run & $\widehat{}$ in-the-house \\
            & $\widehat{}$ with-scissors \\
      \end{tabular}
      \ex{} [\textsubscript{run} run $\widehat{}$ in-the-house]
      $\widehat{}$ with-scissors
      \ex{} [\textsubscript{run} [\textsubscript{run} run $\widehat{}$
      in-the-house] $\widehat{}$ with-scissors]
    \end{xlist}
  }
\end{exe}

Hornstein argues that arguments require labels because, by
stipulation, \textbf{Label} discharges $\theta$-roles and other
argument-relational features.  Because adjuncts do not have such
relational notations, they can be interpreted at the semantics
interface exclusively by their concatenation.  Arguments thus need to
form a complex syntactic atom (and thus must be \textbf{Merge}d), but
adjuncts can simply be \textbf{Concatenate}d.

%%%%%%%%%%%%%%%%%%%%%%%%%%%%%%%%%%%%%%%%%%%%%%%%%%%%%%%%%%%%%%%%%%%%%%%%
\subsection{Hunter's Adjuncts and Arguments}
\label{sec:hunter-adjuncts}

Tim Hunter's 2011 book \textit{Syntactic Effects of Conjunctivist
  Semastics: Unifying movement and adjunction} \cite{hunter2011}
explores adjunction from a semantically-informed view, building on a
modified MG-like syntactic formalism.  Hunter first takes the MG
formalism presented in, e.g. \cite{stabler2001}, with \textbf{Merge}
(in Hunter, \textsc{e-mrg} ``external merge'') and \textbf{Move}
(\textsc{i-mrg}, ``internal/re-merge'').  He then notes that this
formalism adds a great deal of unnecessary structure---namely, once we
have merged together two adjacent constituents, the only information
that is relevant is the phrase as a whole, not the derivation.  Hunter
simplifies \textsc{e-mrg} of adjacent constituents by discarding the
structure and simply concatenating the constituents together.
\textsc{i-mrg} does the same with movers.  As a final step, Hunter
decomposes \textsc{e-mrg} and \textsc{i-mrg} into an insertion
operation, which he calls \textsc{ins} ``insert'' and which inserts a
constituent into the movers of another expression, a feature-checking
operation called \textsc{mrg}, which concatenates adjacent movers, and
a semantic interpretation operation called \textsc{spl} ``spellout''.
This allows Hunter to remove the difference between \textsc{e-mrg} and
\textsc{i-mrg}, and paves the way for Hunter to give an account of
adjuncts.

With a syntactic formalism as a basis, Hunter approaches the question
of adjunction from a specific interpretation of Neo-Davidsonian
logical forms, where each constituent that is merged in denotes a
one-place predicate, and merging results in an \emph{ordered}
conjunction of these predicates to form a new one-place predicate.
Hunter argues that under this view, an adjunct has a denotation that
is already a one-place predicate, and does not need to be modified,
whereas an argument is a multi-place predicate that must be made a
one-place predicate based on its syntactic position.  Thus, an adjunct
can be merged in at different levels in the syntactic structure and
still result in the same semantic interpretation, as long as the order
of the conjuncts in the final logical form is the same.

Hunter then defines a separate set of adjunction features \texttt{*f},
which are able to adjoin to attractors of \texttt{f}.  A lexical item
with a \texttt{*f} feature does not undergo \textsc{mrg}, as it does
not need feature checking and does not need to be made into a
one-place predicate.  An adjunct can thus only undergo \textsc{ins}
and \textsc{spl}, whereas an argument must undergo all three.

%%%%%%%%%%%%%%%%%%%%%%%%%%%%%%%%%%%%%%%%%%%%%%%%%%%%%%%%%%%%%%%%%%%%%%%%
\section{Analysis of Recent Approaches to Adjunction in Minimalism}
\label{sec:analysis}

Fowlie, Hornstein, and Hunter all approach the problem of adjunction
from different perspectives: Fowlie from an MG perspective, Hornstein
from a very Minimalist perspective, and Hunter from a neo-Davidsonian
perspective.  Hornstein and Hunter come up with very similar
approaches, that mirror each other in several ways: they both first
decompose traditional operations into very similar fundamental ones,
they unify \textbf{Merge} and \textbf{Move}, and they conclude that
\textbf{Adjoin} is a simpler operation than either \textbf{Merge} or
\textbf{Move}, which Hunter shows independent evidence for in the
semantics.  Both approaches are attractive, though Hornstein's is more
radical, and satisfy the desiderata we put forth in Section
\ref{sec:adjunction-prop}.  In particular, Hunter's comprehensive
treatment of the semantics of adjuncts provides compelling evidence
for this line of research, and his basis in the MG formalism indicates
that it could still describe the same set of languages as MGs/MCFGs.

Hornstein's approach requires you to fully buy into his framework to
even begin to explain adjuncts, and his framework is rather novel.
Furthermore, his discussion of this framework lacks the formality
necessary to judge it on a computational basis.  Some important
details are not discussed, such as the ordering of multiple adjuncts
to a single target---these would need to be formalized to make strong
claims about both Hornstein's framework and his approach to
adjunction.  Moreover, Hornstein does not make it clear why adjuncts
should be allowed to be labeled at all---if there is no difference in
the semantics, why should extra work be done in some cases, but not
others?  Hunter specifically rejects Hornstein's explanation on this
basis.

Unlike Hornstein and Hunter, Fowlie's approach handles both unordered
and ordered adjuncts, but it does so at a cost: it introduces the need
for counting into the FL, something which many syntacticians have
avoided on theoretical grounds\footnote{See \cite{hornstein2008b} for
  a discussion of this.}.  Fowlie claims that to handle ordered
adjuncts this (or something analogous) is necessary.  In particular,
though, it seems that Fowlie has introduced a generalized
subcategorization means that only appears in a restricted part of the
grammar---\textit{generalized} because it allows an adjunct to select
for a target phrase that has already adjoined to an adjunct of any
subcategory (index) less than it in the hierarchy, and
\textit{restricted} because this is only able to occurs in adjunction.
Compared with Hornstein and Hunter's approaches, where adjunction is a
simpler operation than argument merging, this seems undesired.

On the other hand, Fowlie's model does allow for a more traditional
explanation for explaining the contrast between the examples
(\ref{adjunct-order-normal}--\ref{adjunct-order-marked}), reproduced
here:

\begin{exe}
  \exr{adjunct-order-normal}[]{ Wear the enormous ugly green hat. \\
    \textit{Wear the hat that is enormous, ugly, and green.} }
  \exr{adjunct-order-marked}[\#]{ Wear the ugly enormous green hat. \\
    \textit{Of your enormous green hats, wear the ugly one.} }
\end{exe}

(\ref{adjunct-order-marked}) is not \emph{ungrammatical} here---it's
simply \emph{marked}, and pragmatically anomalous in a neutral
context.  If we assume a $\varnothing_{foc}$ lexical item that
licenses movement within a noun phrase of adjectival adjuncts, we can
capture this in the syntactic structure.  Such a derivation for
(\ref{adjunct-order-marked}) might look as follows:

\begin{exe}
  \ex Wear the [\textsubscript{FocP} ugly $\varnothing_{foc}$
  [\textsubscript{NP} enormous <ugly> green hat]].
\end{exe}

Within an MGA, we could make a lexical item $\epsilon$ : \texttt{=N
  +Afoc [Foc,0,0]} and a lexical item \textit{ugly} : \texttt{[A,2,0]
  -Afoc}; the partial derivation of (\ref{adjunct-order-marked}) would
follow this series of steps:

\begin{enumerate}
\item \textbf{Adjoin} adjacent \textit{green} and \textit{hat} to form
  \texttt{N} at level \texttt{1}.
\item \textbf{Adjoin} nonadjacent \textit{ugly} and \textit{green hat}
  to form \texttt{N} at level \texttt{2} with \textit{ugly} as a
  mover.
\item \textbf{Adjoin} adjacent \textit{enormous} and \textit{green
    hat, ugly} to form \texttt{N} at level \texttt{3} with
  \textit{ugly} as a mover.
\item \textbf{Merge} adjacent $\epsilon$ and \textit{enormous green
    hat, ugly} to form \texttt{+Afoc Foc}.
\item \textbf{Move} \textit{$\epsilon$ enormous green hat, ugly} to
  cancel \texttt{+Afoc} and \texttt{-Afoc} to form \texttt{Foc}.
\item \textbf{Merge} adjacent \textit{the} and \textit{ugly $\epsilon$
    enormous green hat} to form \texttt{D}.
\item \dots
\end{enumerate}

The presence of this $\varnothing_{foc}$ captures the marked nature of
this order.  Unfortunately, such a solution requires duplication in
the lexicon: every adjectival adjunct would need an entry with
\texttt{-Afoc} to be licensed for focus movement and an entry without
\texttt{+Afoc}, to still capture the default order.

Despite this, MGAs show a very important computational result: it is
efficiently recognizition algorithms are possible for adjunction
structures that satisfy all the properties discussed in Section
\ref{sec:adjunction-prop}.  Fowlie shows that MGAs are weakly
equivalent to Multiple Context Free Grammars (MCFGs) \cite[Lemma
7.8.4]{fowlie2015} by introducing a set of transformations from the
fundamental operations \textbf{Merge}, \textbf{Move}, and
\textbf{Adjoin} to MCFG rules; these transformations generate a set of
MCFG rules whose size is polynomial in the number of categorial and
licensing features in the MGA and in the maximum index of any lexical
item in the MGA.  It is known that MCFGs can be parsed in polynomial
time on the length of the string (shown in \cite[Theorem
3.10]{seki1991}, given tighter bounds in \cite{nakanishi1997}); as the
number of MCFG rules generated by a the transformation of an arbitrary
MGA is independent of the length of the string, the recognition
problem for MGAs reduces to the recognition problem for MCFGs, and it
is possible to decide whether a string is generated by a given MGA in
polynomial time.  This finding supports the conclusion that human
language is efficiently parsable while still holding the theoretical
and empirical properties we discussed in Section
\ref{sec:adjunction-prop}.  Such a finding is compelling, as it
indicates that Hunter's MG-informed approach could likely be
efficiently parsable as well; because Hunter's approach is so similar
to Hornstein's, Hornstein's theory is potentially efficiently parsable
as well.  It seems, then, that these three accounts are approaching
something fundamental about adjunction in human language, which fits
in with our current ideas about the computational complexity of human
language.


%%%%%%%%%%%%%%%%%%%%%%%%%%%%%%%%%%%%%%%%%%%%%%%%%%%%%%%%%%%%%%%%%%%%%%%%
\section{Areas of Further Study}
\label{sec:conclusion}

Because of the limited time available to work on this project, we were
only able to research the three accounts of adjunction presented here,
not extend any of them.  Given more time, though, this research
outlines a possible set of next steps that could combine the benefits
of the three approaches.

Because of the strong evidence that Hunter gives, and because he gives
it in a formal enough way that it could be judged on a computational
basis, it seems like an interesting avenue of further study.  Its
ideas are similar to Hornstein's, which were arrived at from a
completely different directly; this lends support to its approach as
well.  As a program of further study, implementing a parser for
Hunter's formalism and then working to extend this with Fowlie's
strictly ordered adjunction strategy would combine the benefits of
both strategies.  For such a combination to work, Hunter's
\textsc{ins} operation would need to check the indices of its two
arguments, and \textsc{mrg} would have to combine them.  Whether this
maintains all the properties of Hunter's formulation is not
immediately clear.

Another possible direction would be to formalize an approach along the
lines of Hornstein.  Such an approach would necessarily be
comprehensive, and it would have to address the problems described
above.  In particular, it would have to wrestle further with the
ordering of multiple adjuncts and come up with an explanation for
\textbf{Label} appears to be optional, neither required nor
disallowed.


%%%%%%%%%%%%%%%%%%%%%%%%%%%%%%%%%%%%%%%%%%%%%%%%%%%%%%%%%%%%%%%%%%%%%%%%
%                                                          BACK MATTER %
%%%%%%%%%%%%%%%%%%%%%%%%%%%%%%%%%%%%%%%%%%%%%%%%%%%%%%%%%%%%%%%%%%%%%%%%


\appendix
%%%%%%%%%%%%%%%%%%%%%%%%%%%%%%%%%%%%%%%%%%%%%%%%%%%%%%%%%%%%%%%%%%%%%%%%
\section{Chart Parsing MGAs: An Implementation Guide to
  \texttt{fowlie-mga}}
\label{sec:implementation}

Just like MGs, MGAs lend themselves nicely to a CKY-like chart parsing
algorithm.  During the research for this project, we implemented a
recognizer based on a chart parser for MGAs to get a greater
understanding of the MGA formalism.  This implementation is
freely-licensed and
\href{https://github.com/pniedzielski/fowlie-mga}{available online as
  \texttt{fowlie-mga}}.  The recognizer is a bottom-up CKY-like
recognizer, and thus views recognizing an input string as performing
deduction on axioms representing the lexical items in the string.


%%%%%%%%%%%%%%%%%%%%%%%%%%%%%%%%%%%%%%%%%%%%%%%%%%%%%%%%%%%%%%%%%%%%%%%%
\subsection{High-Level Design}
\label{sec:implementation-design}

The recognizer is a inference-based chart parser that accepts a string,
tokenizes it by splitting on whitespace, and attempts to deduce the
start symbol of a grammar using the lexical items that appear inside
the string as axioms.

One of the inputs to the recognizer is a grammar that contains a list
of start symbols that should be considered valid end states, a list of
lexical items that appear inside the lexicon, and an association list
that represents the \textbf{Ad} mapping from target categories to
valid adjunct categories. Each lexical item contains an orthographic
representation (normalized to lowercase letters with no whitespace,
and $\epsilon$ if the lexical item is null) and an ordered sequence of
features: \texttt{Category} (\texttt{X}) and \texttt{Selector}
(\texttt{=X}) govern \textbf{Merge}, and \texttt{Licensor}
(\texttt{+X}) and \texttt{Licensee} (\texttt{-X}) govern
\textbf{Move}.  Selectional features additionally carry index
information required to model strictly ordered adjuncts.

The parser functions similarly to that described in
\cite{stabler2001}, with partial parses stored in a \texttt{Chain}
that contains their start and end positions within the input string,
whether they are lexical items or derived syntactic objects, and an
ordered list of features that still have to be checked for the
derivation to succeed.  Each inference rule operates on non-empty
lists of \texttt{Chain}s called \texttt{Expression}s, whose head
represents the head of the parse and whose tail represent ``movers''
that still have to be removed through an application of
\textbf{Move}.  The agenda and chart are unordered lists of these
\texttt{Expression}s, and together form the \texttt{ParseState}.

Before the inference engine is run, we axiomize the agenda and chart
using the \texttt{initialParseState} function, which adds all
recognized lexical items in the string to the chart and agenda, as
well as adds all empty lexical items at all the places they could
occur (that is, before or after any token in the input string).  We
then iteratively infer on this initial \texttt{ParseState} until we
deplete the agenda by calling \texttt{inferAll}.

\texttt{inferAll} generates a lazy list where each element has one
further inference than the last.  It then searches through until it
finds the first element with no more items in the agenda.  Each
inference step along the way generates a list of every possible
conclusion we can form, preferring \textbf{Move} inference rules to
\textbf{Merge} and \textbf{Adjoin}, and selects the first one that we
haven't previously inferred.  If an inference is made, then it is
added to the chart and the agenda.  Otherwise, the item we are trying
to infer on is removed from the agenda, and the next item is tried.
This ensures that the inference engine always either makes forward
progress or terminates.

The functions that model the inference rules were direct
representations of the inference rules shown in Figures
\ref{fig:mg-cky-inference} and \ref{fig:mga-cky-inference}.


%%%%%%%%%%%%%%%%%%%%%%%%%%%%%%%%%%%%%%%%%%%%%%%%%%%%%%%%%%%%%%%%%%%%%%%%
\subsection{Implementation Postmortem}
\label{sec:implementation-postmortem}

Because MGAs are MGs with an additional \textbf{Adjoin} operation, we
decided a good first step would be implementing a recognizer for MG
languages.  Such a recognizer would be less complex, quicker to write,
and easier to understand, all of which make for a good prototype
implementation.  Furthermore, the inference rules for deductively
parsing MGs are well-described online and in the literature (for
example, in \cite{stabler2001}, \cite{harkema2001}, \cite{dost2011},
and \cite{sportiche2013}), and so implementing only an MG recognizer
first would allow me to make use of these resources without additional
complications.  The inference rules for parsing MGs, taken from
\cite{stabler2001} and reproduced in Figure
\ref{fig:mg-cky-inference}, formed the basis for the MG parser.

\begin{figure}[h]
  \centering
  \[
    \inference[merge1]{
      (a, b) :: \text{=}f \gamma &
      (b, c) \cdot f, \alpha_1, \dots, \alpha_k
    }{
      (a, c) : \gamma, \alpha_1, \dots, \alpha_k
    }
  \]
  \[
    \inference[merge2]{
      (b, c) : \text{=}f \gamma, \alpha_1, \dots, \alpha_k &
      (a, b) \cdot f, \iota_1, \dots, \iota_l
    }{
      (a, c) : \gamma, \alpha_1, \dots, \alpha_k, \iota_1, \dots, \iota_l
    }
  \]
  \[
    \inference[merge3]{
      (a, b) \cdot \text{=}f \gamma, \alpha_1, \dots, \alpha_k &
      (c, d) \cdot f \delta, \iota_1, \dots, \iota_l
    }{
      (a, b) : \gamma, \alpha_1, \dots, \alpha_k, (c, d) : \delta, \iota_1, \dots, \iota_l
    }
  \]
  \[
    \inference[move1]{
      (b, c) : +g \gamma, \alpha_1, \dots, \alpha_{i-1}
      (a, b) : -g, \alpha_{i+1}, \dots, \alpha_k
    }{
      (a, c) : \gamma, \alpha_1, \dots, \alpha_{i-1}, \alpha_{i+1}, \dots, \alpha_k
    }
  \]
  \[
    \inference[move2]{
      (a, b) : +g \gamma, \alpha_1, \dots, \alpha_{i-1}
      (c, d) : -g \delta, \alpha_{i+1}, \dots, \alpha_k
    }{
      (a, b) : \gamma, \alpha_1, \dots, \alpha_{i-1}, (c, d) : \delta, \alpha_{i+1}, \dots, \alpha_k
    }
  \]
  \caption{CKY inference rules for MGs.  These inference rules are
    defined with an input string of length $N$ on any positions
    $0 \leq a,b,c,d \leq N$, types $\cdot \in \{ :, :: \}$, categorial
    features $\{f\}$, licensing features $\{g\}$, possibly empty
    sequences of features $\gamma, \delta$, and chains
    $\alpha_1, \dots, \alpha_k; \iota_1,\dots,\iota_l$ for
    $k, l \geq 0$.}
  \label{fig:mg-cky-inference}
\end{figure}

Implementing the a deductive chart parser was the most challenging
part of implementing the MGA recognizer, and during the MG phase of
the project, we had several false starts that turned out to be the
wrong way to go.  In particular, these stemmed from a misunderstanding
of the role of the agenda in a deductive chart parser: because the
agenda is implicit in the standard CKY algorithm and is captured in
the way the algorithm iterates through the chart, we first thought an
MG chart parser could play a similar trick, simply iterating through
the chart in the right order to check all possible inferences.  The
first implementation tried this and did not produce the correct
output; it wasn't until we formalized the loop invariants in the
CKY-like algorithm we were using that we found this was the source of
the error, and we implemented it using a separate, explicit agenda
data structure.  Later, we found we were placing results in the
incorrect location in our two-dimensional CKY-like chart.  Such a
chart makes less sense in an MG recognizer with movers, and it was
less intuitive to use in an MG recognizer than it had been in our CFG
recognizer.  We decided to abandon this chart in favor of a simple
list data structure representing a set.  Although this added an
additional linear search to every lookup, it allowed us to use
\texttt{Data.List} module functionality and the list monad, which made
the code clearer and easier to work with.  As performance was not our
goal, our grammars were simple, and our input strings were short, we
found this to be a reasonable trade-off.  A data structure with
sub-linear lookup could be used if this were found to be unreasonably
slow, or the more classical approach of considering the chart a graph
could be considered.

Once we had a working MG recognizer, we modified the code to make it
recognize MGAs.  There we three parts to this modification: we needed
to add the inference rules required to implement \textbf{Adjoin}, we
needed to modify the chart entries and grammars to contain indices
needed by \textbf{Adjoin}, and we needed to modify inference
application to apply the new \textbf{Adjoin} operations.  The
inference rules, shown below in Figure \ref{fig:mga-cky-inference},
were created by analogy with those of the the \textbf{Merge}
operation, to which they are conceptually quite similar.  There is no
need for an inference rule analogous to merge1 (the complement merging
inference rule), because by stipulation, we order all adjuncts before
their heads, as per \cite{fowlie2015}.  Implementing these was
straight-forward.

\begin{figure}[h]
  \centering
  % \[
  %   \inference[merge1]{
  %     (a, b) :: \text{=}f \gamma &
  %     (b, c) \cdot [f,i',j'], \alpha_1, \dots, \alpha_k
  %   }{
  %     (a, c) : \gamma, \alpha_1, \dots, \alpha_k
  %   }
  % \]
  % \[
  %   \inference[merge2]{
  %     (b, c) : \text{=}f \gamma, \alpha_1, \dots, \alpha_k &
  %     (a, b) \cdot [f,i,j], \iota_1, \dots, \iota_l
  %   }{
  %     (a, c) : \gamma, \alpha_1, \dots, \alpha_k, \iota_1, \dots, \iota_l
  %   }
  % \]
  % \[
  %   \inference[merge3]{
  %     (a, b) \cdot \text{=}f \gamma, \alpha_1, \dots, \alpha_k &
  %     (c, d) \cdot [f,i,j] \delta, \iota_1, \dots, \iota_l
  %   }{
  %     (a, b) : \gamma, \alpha_1, \dots, \alpha_k, (c, d) : \delta, \iota_1, \dots, \iota_l
  %   }
  % \]
  % \[
  %   \inference[move1]{
  %     (b, c) : +g \gamma, \alpha_1, \dots, \alpha_{i-1}
  %     (a, b) : -g, \alpha_{i+1}, \dots, \alpha_k
  %   }{
  %     (a, c) : \gamma, \alpha_1, \dots, \alpha_{i-1}, \alpha_{i+1}, \dots, \alpha_k
  %   }
  % \]
  % \[
  %   \inference[move2]{
  %     (a, b) : +g \gamma, \alpha_1, \dots, \alpha_{i-1}
  %     (c, d) : -g \delta, \alpha_{i+1}, \dots, \alpha_k
  %   }{
  %     (a, b) : \gamma, \alpha_1, \dots, \alpha_{i-1}, (c, d) : \delta, \alpha_{i+1}, \dots, \alpha_k
  %   }
  % \]
  \[
    \inference[adjoin1]{
      (a, b) \cdot [f,i,j] &
      (b, c) \cdot [f',i',j'] \gamma, \alpha_1, \dots, \alpha_k
    }{
      (a, c) : [f', i', i]\gamma, \alpha_1, \dots, \alpha_k
    }[iff $f \in \mathbf{Ad}(f')$ and $i \geq j'$]
  \]
  \[
    \inference[adjoin2]{
      (a, b) \cdot [f,i,j] \gamma &
      (c, d) \cdot [f',i',j'] \delta, \alpha_1, \dots, \alpha_k
    }{
      (c, d) : [f', i', i]\delta, (a, b) : \gamma, \alpha_1, \dots, \alpha_k
    }[iff $f \in \mathbf{Ad}(f')$ and $i \geq j'$]
  \]
  \caption{Additional CKY inference rules for MGAs.  The same
    inference rules in \ref{fig:mg-cky-inference} are used for
    \textbf{Merge} and \textbf{Move}, and these additional adjoin1 and
    adjoin2 rules are modelled after the \textbf{Merge} rules.}
  \label{fig:mga-cky-inference}
\end{figure}

Because of the roundabout way we arrived at current implementation of
the inference engine, it required small modifications to support MGAs:
inference rules are hard-coded instead of passed in to the engine,
types are hard-coded instead of abstracted away through typeclasses,
and merge and move had to be changed when the grammar changed,
indicating that we had the incorrect level of abstraction.  A
refactoring that separates the inference engine from any specifics of
the MGs or MGAs and that separates the inference rules from the
representation of the grammar would make the code cleaner and easier
to understand.  It would prevent us from using pattern matching in the
inference rules, though; this could still be abstracted away as
functions that the grammar exposes.

Initially, we had planned to implement a complete parser that would
return either a parse tree or an ordered list of \textbf{Move},
\textbf{Merge}, and \textbf{Adjoin} operations that generate the given
input, not just a recognizer.  This would not have been too much
additional work over the recognizer, but we felt that it would not
help in our goal of gaining a greater understanding of MGAs as a
formalism.  Indeed, transforming the recognizer into a parser would
not have resulted in any changes to the MGA-specific part of the
code base, only to the inference engine.


%%%%%%%%%%%%%%%%%%%%%%%%%%%%%%%%%%%%%%%%%%%%%%%%%%%%%%%%%%%%%%%%%%%%%%%%
%                                                          BACK MATTER %
%%%%%%%%%%%%%%%%%%%%%%%%%%%%%%%%%%%%%%%%%%%%%%%%%%%%%%%%%%%%%%%%%%%%%%%%


\cleardoublepage
\phantomsection
\addcontentsline{toc}{section}{References}
%\nocite{*}
\bibliographystyle{alpha}
\bibliography{writeup}

\end{document}

%%% Local Variables:
%%% mode: latex
%%% ispell-dictionary: "english"
%%% TeX-master: t
%%% End:
